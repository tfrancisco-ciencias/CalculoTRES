\documentclass[12pt]{article}
\setlength{\topmargin}{-.75in}\addtolength{\textheight}{2.00in}
\setlength{\oddsidemargin}{.00in}\addtolength{\textwidth}{.75in}
\usepackage{amsmath,amssymb}
\nofiles

\pagestyle{empty}
\setlength{\parindent}{0in}
\newcommand{\sen}{\operatorname{sen}}


\begin{document}

\noindent{\sc {\bf{\large Quiz 10 }}
           \hfill Cálculo 3, semestre 2020-2}
\bigskip

\noindent {\sc{
            { \Large Nombre: \underline {\hspace {10 cm }}}}}
            
\bigskip
\bigskip
\bigskip


<ol>

  
\item(2 pts) Para las siguientes funciones $f:\mathbb{R}^3 \to \mathbb{R}$ y
  $\gamma:\mathbb{R}\to \mathbb{R}^3$, encuentra $\nabla f $, $\gamma'(t)$ y usa la regla de la
  cadena para calcular $(f\circ \gamma)'(1)$.
  $$f(x,y,z)=xz+yz+xy, \quad \gamma(t)=(e^t,\cos(t),\sen(t))$$
 
 


  \vspace{3cm}

\item(3 pts) Usa la regla de la cadena para calcular $\partial_sh(1,0)$, donde
  $h(s,t)=f(u(s,t),v(s,t))$ donde
  \begin{eqnarray*}
    f(u,v)&=&\cos(u)\sen(v)\\
    u(s,t)&=&\cos(t^2s),\quad v(s,t)= \log(\sqrt{1+s^2})
  \end{eqnarray*}
  
</ol>


  
\end{document}
