\documentclass[12pt]{article}
\setlength{\topmargin}{-.75in}\addtolength{\textheight}{2.00in}
\setlength{\oddsidemargin}{.00in}\addtolength{\textwidth}{.75in}
\usepackage{amsmath,amssymb}
\nofiles

\pagestyle{empty}
\setlength{\parindent}{0in}
\newcommand{\sen}{\operatorname{sen}}


\begin{document}

\noindent{\sc {\bf{\large Quiz 11 }}
           \hfill Cálculo 3, semestre 2020-2}
\bigskip

\noindent {\sc{
            { \Large Nombre: \underline {\hspace {10 cm }}}}}
            
\bigskip
\bigskip
\bigskip


<ol>

  
<div class="ejercicio-box"> <h2 class="number-title"> Ejercicio</h2>(2 pts) Halla una normal unitaria a la superficie: $\cos(xy)=e^z-2$, en $(1,\pi,0)$.
 


  \vspace{3cm}

<div class="ejercicio-box"> <h2 class="number-title"> Ejercicio</h2> Considera las funciones $F:\mathbb{R}^2 \to \mathbb{R}^2$ y $G:\mathbb{R}^3 \to \mathbb{R}^2$
  dadas por
  $$
  F(x,y)=(e^{x+2y}, \sen(y+2x)), \quad G(u,v,w)=(u+2v^2+3w^3, 2v-u^2)
  $$

  <ol>
  <div class="ejercicio-box"> <h2 class="number-title"> Ejercicio</h2> (1.5 pts)Calcula las matrices de derivadas parciales $D_{(x.y)}F$ y $D_{(u,v,w)}G$.
  <div class="ejercicio-box"> <h2 class="number-title"> Ejercicio</h2> (1.5 pts)  Usa la regla de la cadena para encontar la matriz de derivadas
    parciales  $D_{(1,-1,1)}F\circ G$.
  </ol>
  
</ol>


  
\end{document}
