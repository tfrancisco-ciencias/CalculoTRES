
\documentclass{article}
\usepackage{amssymb, latexsym, amsmath, amsthm, amsfonts, amsbsy, enumerate, tabularx, graphicx}
\newtheorem{definicion}{Definición}

\newtheorem{teorema}{Teorema}
\newtheorem{prop}{Proposición}
\newtheorem{coro}{Corolario}
\newtheorem{lema}{Lema}
\theoremstyle{definition}
\newtheorem{nota}{Nota}
\newtheorem{notas}{Notas}
\newtheorem{ejemplo}{Ejemplo}
\newtheorem{ejemplos}{Ejemplos}
%\newtheorem{definicion}{Definición}
\newtheorem{notacion}{Notación}
\newcommand{\sen}{\operatorname{sen}}
\newcommand{\senh}{\operatorname{senh}}
\newcommand{\csch}{\operatorname{csch}}
\newcommand{\sech}{\operatorname{sech}}
\newcommand{\arcsec}{\operatorname{arcsec}}
\newcommand{\arcsen}{\operatorname{arcsen}}
\begin{document}

<ol>


  \section*{T0}

\section*{Vectores}


    
 \item Soluciona la siguiente ecuación para $x,y,z$.
    <ol>
    \item
      $$
      x(1,2,3)+(y,3y,2)+(z,2,1)=(1,2,3).
     $$
    \item
      $$
     y(-1,3,2)+(x,-x,3)+(z,2,5)=(-1,0,3).
      $$
    </ol>

  \item Encuentra todos los pares de números reales $a,b$ que satisfacen
    $$
    a(-1,1)+b(1,1)=0.
    $$



  \item Encuentra las ecuaciones de la rectas con la información dada
    <ol>
    \item La recta que pasa por el punto $(1,2,3)$ con vector dirección $v=(-1,2,-3)$.
    \item La recta que pasa por el punto $(5,10,15)$ y que es paralela al eje $x$.
      \item La recta que por el punto $(3,-2,-3)$ y que es perpendicular al plano $yz$.
    </ol>
    
    \item Encuentra todos los puntos de intersección de la recta $x=2+5t,y=5-2t,z=6+3t$ con los planos coordenados.

      
    \item Determina si las rectas con ecuaciones paramátricas
      $$
      x=1+6t, \quad y=3-3t, \quad z=3t-2, \quad t\in \mathbb{R}
      $$
      
      $$
      x=2-5s, \quad y=1-2s,\quad z=1+s, \quad s \in \mathbb{R}
      $$
      se intersectan o no.
 

    \item Determina si los puntos dados están  o no en une misma recta:
      <ol>
      \item $(2,2,4)$, $(1,2,5)$, $(1,2-5)$.
      \item $(1,-1,-2)$, $(3,0,1)$, $(5,-1,0)$
      </ol>
          
       
       
       \section{Producto interior, norma}


	<div class="definicion-box"><h2 class="number-title">  Definición</h2>
	En $\mathbb{R}^n$ se define la norma de un vector $p=(p_1,\dots, p_k)$ como
	$$
		\|p\|=\left( \sum_{k=1}^k p_k\right)^{1/2}
	$$
	
	La distancia entre dos puntos $p,q \in \mathbb{R}^n$ se define como
	$$
	\|p-q\|	
	$$
	</div>
       
       \item Describe el conjunto de puntos $p=(x,y)$ en el  plano que satisface
         <ol>
         \item $\|p\|_2=3$
         \item $\|p-\hat{j}\|_2=6$
         \item $ \langle p, \hat{i} \rangle=2$
         \item $\langle p, \hat{i} \rangle = \|p\|_2$
         </ol>
         Recuerda que $\hat{i}=(1,0)$ y $\hat{j}=(0,1)$.



       \item Sean $v, w\in \mathbb{R}^n$.
       	Usando las propiedades del procuto interior, demostrar la siguiente identidad (llamada ley del paralelogramo)
         $$
         \|v+w\|^2+\|v-w\|^2=2(\|v\|^2+\|w\|^2)
         $$


        
			\item Para un vector $p=(p_1\dots, p_n) $ demuestra
			<ol>
			\item $	\|p\| \leq \sum_{i=1}^n|p_i| $. 
			\item para toda $i=1,\dots, n$, $|p_i| \leq \|p\|$
			</ol>
                
       \item Usar la dsigualdad de Cauchy-Schwarz para probar que, para cualesqueira tres
         reales $a,b,c$
         $$
         (a+b+c)^2 \leq 3(a^2+b^2+c^2)
         $$
         

       </ol>


  
       \end{document}
