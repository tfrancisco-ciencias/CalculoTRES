
\documentclass{article}
\usepackage{amssymb, latexsym, amsmath, amsthm, amsfonts, amsbsy, enumerate, tabularx, graphicx}
\newtheorem{definicion}{Definición}

\newtheorem{teorema}{Teorema}
\newtheorem{prop}{Proposición}
\newtheorem{coro}{Corolario}
\newtheorem{lema}{Lema}
\theoremstyle{definition}
\newtheorem{nota}{Nota}
\newtheorem{notas}{Notas}
\newtheorem{ejemplo}{Ejemplo}
\newtheorem{ejemplos}{Ejemplos}
%\newtheorem{definicion}{Definición}
\newtheorem{notacion}{Notación}
\newcommand{\sen}{\operatorname{sen}}
\newcommand{\senh}{\operatorname{senh}}
\newcommand{\csch}{\operatorname{csch}}
\newcommand{\sech}{\operatorname{sech}}
\newcommand{\arcsec}{\operatorname{arcsec}}
\newcommand{\arcsen}{\operatorname{arcsen}}
\newcommand{\id}{\operatorname{id}}
\begin{document}




\section*{T8 }

<ol>
\item  Sea $U\ne \emptyset$ un abierto de $\mathbb{R}^n$, $p_0\in U$, $f,g :U \to \mathbb{R}$
  funciones diferenciables en $p_0$ y  $c\in \mathbb{R}$.

  Usando la proposición 2 de las notas,  demuestra que $cf+g$ es diferenciable
  en $p_0$ y que:
  $$
  \nabla_{p_0}(cf+g)=c\nabla_{p_0} f+ \nabla_{p_0}g.
  $$

\item Sea $f:\mathbb{R}^n\to \mathbb{R}$
  una función diferenciable en todo punto de $\mathbb{R}^n$ ¿  Tiene sentido la fórmula
  $$
  \nabla_{p_0+p_0'}f=\nabla_{p_0}f+\nabla_{p_0'}f ?
  $$

  Demuestra o da un contraejemplo.

\item Encuentra la aproximación lineal de la función indicada, alrededor de un punto adecuado
  para estimar  las siguientes cantidades
  <ol>
  \item $\sqrt{(9.1)(15.9)}$,
  \item $\sqrt{(4.1)^2+(3.95)^2+(2.01)^2}$,
  \item $(2.01)^3+(1.99)^3-5(2.01)(1.99)$.
  </ol>

  
          \item\label{Ejer:DifPolinomios}
            Este ejercicio muestra que todo polinomio en dos variables es diferenciable en todo
            punto de $\mathbb{R}^2$.

            <ol>
            \item Dados naturales $n,m\geq 0$ considera la función $f:\mathbb{R}^2\to \mathbb{R}$
              dada por $f(x,y)=x^ny^m$. Demuestra que $f$ es diferenciable en todo punto de $\mathbb{R}^2$.  Sugerencia: usa el ejercicio 2 de aproximación
              lineal (el del video). 

             

            \item Un polinomio en dos variables en una función de la forma
              $$
              p(x,y)= \sum_{i=1}^N \alpha_i x^{n_i}y^{m_i}
              $$
              donde, para $i=1,\dots, N$,  $\alpha_i\in \mathbb{R}$ y $n_i,m_i\geq 0$ son naturales.

              Demuestra que $p$ es diferenciable en todo punto de $\mathbb{R}^2$.

              
              Nota: también se puede probar que todo polinomio en $n$-variables es diferenciable.
              
            \item Considera el polinomio en dos variables
              $$p(x,y)= a + bx+ cy + \sum_{i=1}^N x^{i}y^{i+1}$$
              donde $a,b,c\in \mathbb{R}$ y $N\geq 1$ es un natural.
              
              Encuentra su aproximación lineal de $p$ cerca del $(0,0)$.

              Sugerencia: primero prueba que, $\lim_{(x,y)\to (0,0)} \frac{|x^ny^m|}{\|(x,y)\|}=0$ si
              $n,m\geq0$ son naturales con $n+m\geq 2$ y después usa la proposición 2 de las notas.
            </ol>

            
             \item Considera  la función  $r:\mathbb{R}^3 \to \mathbb{R}$ dada por $r(x,y,z)=\|(x,y,z)\|$. Durante
            este ejercicio vamos a suponer que $r$ es difeferenciable en todo punto $(x_0,y_0,z_0)\ne (0,0,0)$.

            Por simplicidad vamos a denotar: $p_0=(x_0,y_0,z_0)$.
            <ol>
            \item Demuestra que $\nabla_{p_0}r$ es el  vector unitario, en la misma dirección que
              $p_0$.
            \item Demuestra que $\nabla_{p_0}(r^n)=[nr^{(n-2)}(x_0)]p_0$, donde $n \geq 1$ es un natural.
            \item La fórmula del inciso anterio ¿ es válida si $n$ es un entero negativo?
            </ol>
         
  
          \item Sea $q_0$ un vector fijo en $\mathbb{R}^n$ y define $f:\mathbb{R}^n \to \mathbb{R}$
            por $f(p)=\langle q_0, p \rangle$. Demuestra que $f$ es continua en todo punto $p \in \mathbb{R}^n$.
           \item\label{Ejer:AproxLinealReglaCadena} 
            Sea $U\ne \emptyset$ un abierto de $\mathbb{R}^n$, $p_0\in U $ y $g:U \to \mathbb{R}$
            una función diferenciable en $p_0$.
            <ol>
            \item Demuestra que existe una bola $B_r(p_0)$, una función
              $F:B_{r}(p_0) \to \mathbb{R}$ tal que
              <ol>
              \item para toda $p\in B_r(p_0)$, $g(p)=g(p_0)+\langle \nabla_{p_0}g, p-p_0\rangle +\|p-p_0\|F(p)$,
              \item $\lim_{p\to p_0}F(p)=0$,
              </ol>
              Sugerencia: si $E$ es la función error de la aproximación de $g$ en $p_0$ define:
              $$
              F(p)=\left\{
                \begin{array}{cc}
                  \frac{E(p)}{\|p-p_0\|} & p\ne p_0 \\
                  0 & p=p_0.
                \end{array}
              \right.
              $$

            \item Demuestra que existe una bola $B_s(p_0)$ tal que para toda $p\in B_s(p_0)$,
              $\|F(p)\|\leq 1/2$.

              Sugerencia: en la definición $\varepsilon,\delta$ del límite $\lim_{p\to p_0}F(p)=0$
              toma $\varepsilon =1/2$.
              
            \item Demuestra que  para toda $p\in B_s(p_0)$
              $$
              |g(p)-g(p_0)|\leq \|p-p_0\| (\|\nabla_{p_0}g\|+1/2).
              $$

              Sugerencia: usa el inciso anterior y la desigualdad de Cacuchy-Schwartz.
              
            </ol>

            \item  Considera la función  $t\mapsto 1/t$, definida para $t\in \mathbb{R}, t\ne 0$.

              <ol>
                \item Prueba que,  dado $t_0\ne 0$ fijo y arbitrario, existe $s>0$ tal que para toda $t\in B_s(t_0)$  
              $$
              \frac{1}{t}=\frac{1}{t_0}-\frac{1}{t_0^2}(t-t_0)+|t-t_0|F_1(t)
              $$
              donde $\lim_{t\to t_0} |F_1(t)| =0$.
            \item Sean $a,b\in \{1,2,3,4,5,6,7,8,9\}$ tal que $a+b=10$. Usa el inciso anterior para
              probar que el inverso multiplicativo de $1.a$ es aproximadamente $0.b$.
            </ol>
            
          %\item Sea $U$ un abierto de $\mathbb{R}^n$  y $f,g:U\to \mathbb{R}$ dos funciones.
           % Sea $p_0 \in U$ un punto tal que tanto $f$ como $g$ son diferenciables en $p_0$.
            %Asume que $g(p_0)\ne 0$.
          
           %Demuestra que la función $f/g$ es diferenciable en $p_0$ y que
            %  $$
             % \nabla_{p_0}(f/g)=\frac{g(p_0)\nabla_{p_0}f - f(p_0)\nabla_{p_0}g}{g(p_0)^2}
              %$$

         
            
</ol>


  
       \end{document}
