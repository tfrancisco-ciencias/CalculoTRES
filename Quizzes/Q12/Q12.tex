\documentclass[12pt]{article}
\setlength{\topmargin}{-.75in}\addtolength{\textheight}{2.00in}
\setlength{\oddsidemargin}{.00in}\addtolength{\textwidth}{.75in}
\usepackage{amsmath,amssymb}
\nofiles

\pagestyle{empty}
\setlength{\parindent}{0in}
\newcommand{\sen}{\operatorname{sen}}


\begin{document}

\noindent{\sc {\bf{\large Quiz 12 }}
           \hfill Cálculo 3, semestre 2020-2}
\bigskip

\noindent {\sc{
            { \Large Nombre: \underline {\hspace {10 cm }}}}}
            
\bigskip
\bigskip
\bigskip


Considera la función $f(x,y)=(x-y)(xy-1)$.



<ol>

<div class="ejercicio-box"> <h2 class="number-title"> Ejercicio</h2> (3pts) Determina la fórmula de Taylor, de orden 2, para $f$
  en el punto $(\sqrt{3},1)$.



  \vspace{8cm}


<div class="ejercicio-box"> <h2 class="number-title"> Ejercicio</h2> (2pts) Determina si el Hessiano de $f$ en $(\sqrt{3},1)$ es
  definitivamente positivo, definitivamente negativo o ninguno.


  

</ol>
 





  
\end{document}
