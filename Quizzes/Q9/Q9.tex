\documentclass[12pt]{article}
\setlength{\topmargin}{-.75in}\addtolength{\textheight}{2.00in}
\setlength{\oddsidemargin}{.00in}\addtolength{\textwidth}{.75in}
\usepackage{amsmath,amssymb}
\nofiles

\pagestyle{empty}
\setlength{\parindent}{0in}
\newcommand{\sen}{\operatorname{sen}}


\begin{document}

\noindent{\sc {\bf{\large Quiz 9 }}
           \hfill Cálculo 3, semestre 2020-2}
\bigskip

\noindent {\sc{
            { \Large Nombre: \underline {\hspace {10 cm }}}}}
            
\bigskip
\bigskip
\bigskip


<ol>

  
\item (2 pts) Considera la función
  $f(x,y)=e^{-x^2+y+1}$, $(x,y)\in \mathbb{R}$. Encuentra:
  <ol>
    \item la dirección $u$ ($\|u\|=1$) para la cual
      la razón de crecimiento de $f$ es la más grande si estás
      en el punto $(2,3)$;
     \item una dirección $v$ ($\|v\|=1$), para la cual
       la razón de crecimiento de $f$ es cero, si estás en el punto
       $(2,3)$.
</ol>


  
\vspace{4cm}  


\item (3 pts) Encuentra una función $f(x,y)$, cuyo gradiente es:
  $$(2\cos(2x)e^{\sen(2x)+\cos(y^2)},-2y\sen(y^2)e^{\sen(2x)+\cos(y^2)})$$

</ol>


  
\end{document}
