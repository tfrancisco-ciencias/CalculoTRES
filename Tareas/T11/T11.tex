
\documentclass{article}
\usepackage{amssymb, latexsym, amsmath, amsthm, amsfonts, amsbsy, enumerate, tabularx, graphicx}
\newtheorem{definicion}{Definición}

\newtheorem{teorema}{Teorema}
\newtheorem{prop}{Proposición}
\newtheorem{coro}{Corolario}
\newtheorem{lema}{Lema}
\theoremstyle{definition}
\newtheorem{nota}{Nota}
\newtheorem{notas}{Notas}
\newtheorem{ejemplo}{Ejemplo}
\newtheorem{ejemplos}{Ejemplos}
%\newtheorem{definicion}{Definición}
\newtheorem{notacion}{Notación}
\newcommand{\sen}{\operatorname{sen}}
\newcommand{\senh}{\operatorname{senh}}
\newcommand{\csch}{\operatorname{csch}}
\newcommand{\sech}{\operatorname{sech}}
\newcommand{\arcsec}{\operatorname{arcsec}}
\newcommand{\arcsen}{\operatorname{arcsen}}
\newcommand{\id}{\operatorname{id}}
\begin{document}




\section*{Tarea 11 }


<ol>


  <div class="ejercicio-box"> <h2 class="number-title"> Ejercicio</h2> Supon que $I\subset \mathbb{R}$ es un intervalo abierto
    y que $\gamma,\beta :I\to \mathbb{R}^2$  son curvas diferenciables.
    <ol>
      <div class="ejercicio-box"> <h2 class="number-title"> Ejercicio</h2> Demuestra
        $$
        \frac{d}{dt} \langle \beta(t), \gamma(t) \rangle
        =\langle \beta(t), \gamma'(t) \rangle
        + \langle \beta'(t) , \gamma(t) \rangle
        $$
      <div class="ejercicio-box"> <h2 class="number-title"> Ejercicio</h2> Ahora supon que la imagen de $\gamma$ está contenida
        en una esfera centrada en el origen. Usa el inciso anterior
        para demostrar que, para toda $t$,
        $\langle \gamma(t), \gamma'(t)\rangle=0$, es decir
        el vector posición y el vector velocidad son ortogonales.

      </ol>
      
  <div class="ejercicio-box"> <h2 class="number-title"> Ejercicio</h2> Usa la regla de la cadena para escribir las derivadas 
    parciales que se piden en las siguientes funciones:

    <ol>
    <div class="ejercicio-box"> <h2 class="number-title"> Ejercicio</h2> $\frac{\partial f}{\partial y}(A(x,y),B(x,y),C(x,y))$,
    <div class="ejercicio-box"> <h2 class="number-title"> Ejercicio</h2> $\frac{\partial f}{\partial x_2}(p_1(x_1,x_2,x_3),p_2(x_1,x_2,x_3))$
    <div class="ejercicio-box"> <h2 class="number-title"> Ejercicio</h2> $\frac{\partial f}{\partial x_5}
      (X_1(x_1, \dots, x_{10}), \dots, X_{7}(x_1,\dots, x_{10}))$
    </ol>

<div class="ejercicio-box"> <h2 class="number-title"> Ejercicio</h2> Sea $U\ne \emptyset$ un abierto de
  $\mathbb{R}^2$, $g:U\to \mathbb{R}$ una función de clase $C^1$
  y $x,y:I\to \mathbb{R}$ funciones  de clase $C^1$
  (con $I$ un intervalo abierto).
  Encuentra $h'(t)$, la derivada de
  $h$ (función de una variable), para las siguientes funciones:
  <ol>
  <div class="ejercicio-box"> <h2 class="number-title"> Ejercicio</h2> $h(t)=(g(x(t), y(t)))^2$,
  <div class="ejercicio-box"> <h2 class="number-title"> Ejercicio</h2> $h(t)=e^{g(x(t),y(t))}$,
  <div class="ejercicio-box"> <h2 class="number-title"> Ejercicio</h2> $h(t)=\sen(g(x(t),y(t)))$.
  </ol>


<div class="ejercicio-box"> <h2 class="number-title"> Ejercicio</h2> Se dice que una función $f$ tiene rendimientos de escala constante,
  o que es homogenea de grado 1, si satisface
  \begin{equation}\label{Eqn:RendimientoEscala}
  f(cx,cy)=cf(x,y)
  \end{equation}
  para toda $c>0$. El nombre viene porque, por ejemplo,
  si se duplica $x$ y $y$, también se duplica $f$.

  <ol>
  <div class="ejercicio-box"> <h2 class="number-title"> Ejercicio</h2> Prueba que $f(x,y)=\sqrt{x^2+y^2}$ y  $f(x,y)=\sqrt{xy}$, $x,y>0$,
    tienen rendimientos de escala constante.
  <div class="ejercicio-box"> <h2 class="number-title"> Ejercicio</h2> Demuestra que si $f$ es clase $C^1$ y tiene rendimientos
    de escala constante entonces $f$ satisface:
    \begin{equation}\label{Eqn:RendimientoEscala2}
    x \partial_x f(x,y) + y \partial_y f(x,y)=f(x,y)
    \end{equation}

    Sugerencia: usa la regla de la cadena para
    diferenciar la ecuación \eqref{Eqn:RendimientoEscala}
    con respecto a $c$ y luego evalua en $c=1$.

  <div class="ejercicio-box"> <h2 class="number-title"> Ejercicio</h2> Demuestra directamente que $f(x,y)=\sqrt{x^2+y^2}$ y $f(x,y)=\sqrt{xy}$
    satisfacen la ecuacion \eqref{Eqn:RendimientoEscala2}.

  <div class="ejercicio-box"> <h2 class="number-title"> Ejercicio</h2> Encuentra un ejemplo de una función de rendimiento de escala
    constante diferente a las dadas en el ejercicio.
  </ol>
  
<div class="ejercicio-box"> <h2 class="number-title"> Ejercicio</h2> Un alambre tiene forma circular, de radio $r$ y supongamos que está
  colocado con su centro en el origen. Supongamos que
  $T:\mathbb{R}^2 \to \mathbb{R}$ es la función que da la temperatura en
  el punto $(x,y)$ del plano. Para las siguientes funciones
  temperatura, encuentra las temperaturas máximas y mínimas
  que sufre el alambre.

  <ol>
  <div class="ejercicio-box"> <h2 class="number-title"> Ejercicio</h2> $T(x,y)=2x+3y$,
  <div class="ejercicio-box"> <h2 class="number-title"> Ejercicio</h2> $T(x,y)=2xy$,
  <div class="ejercicio-box"> <h2 class="number-title"> Ejercicio</h2> $T(x,y)=x^2+y^2$.
  </ol>

  Sugerencia: empieza dando $\gamma$, una parametrización del alambre,
  luego encuentra el máximo y el mínimo de $T\circ \gamma$.
  
  
<div class="ejercicio-box"> <h2 class="number-title"> Ejercicio</h2> Una función $f:\mathbb{R}^n \to \mathbb{R}$ se llama par si
satisface $f(-p)=f(p)$, para todo punto  $p$. Supon que $f$ es clase
$C^1$ y demuestra que  $\nabla_{(0,\dots, 0)}f=(0,\dots, 0)$.

Sugerencia: aplica la regla de la cadena a $f(p)=f(-p)$. 



<div class="ejercicio-box"> <h2 class="number-title"> Ejercicio</h2> Sea $y(x)$ una función definida implícitamente por $G(x,y(x))=0$,
  donde $G$ es una función de clase $C^1$ definida en $\mathbb{R}^2$.
  Prueba que si $y$ es de clase
  $C^1$  y  $\frac{\partial G}{\partial y}\ne 0$ entonces
  $$
  \frac{dy}{dx}=-\frac{ \partial G/ \partial x}{\partial G/\partial y}, 
  $$



<div class="ejercicio-box"> <h2 class="number-title"> Ejercicio</h2> Prueba los siguientes pasos para demostrar la  fórmula:
  \begin{equation}\label{Eqn:FormulaDerivadaInt}
  \frac{d}{dt} \int_{y_1(t)}^{y_2(t)}g(x,t)dx=
  \int_{y_1(t)}^{y_2(t)}\frac{\partial g(x,t)}{\partial t}dx+ g(y_2(t),t)y_2'(t)
  -g(y_1(t),t)y_1'(t)
  \end{equation}

  donde $g(x,t)$ es una función clase $C^1$ en un abierto $U$
  y $y_1,y_2$ funciones de clase $C^1$ de una variable.

  <ol>
  <div class="ejercicio-box"> <h2 class="number-title"> Ejercicio</h2> Define $f(u,v,w)=\int_u^v g(x,w)dx$. Usa el Teorema fundamental
    del cálculo para probar que
    $$
    \frac{\partial f}{\partial u}(u,v,w)=-g(u,w),
    \quad \frac{\partial f}{\partial v}(u,v,w)=g(v,w).
    $$

  <div class="ejercicio-box"> <h2 class="number-title"> Ejercicio</h2> Usa la regla de la cadena para probar
    $$
    \frac{d}{dt} f(y_1(t),y_2(t),t)= \frac{\partial f}{\partial u}
    y_1'(t)+ \frac{\partial f}{\partial v}y_2'(t)
    +\frac{\partial f}{\partial w}
    $$

  <div class="ejercicio-box"> <h2 class="number-title"> Ejercicio</h2> Se puede probar, no lo demuestres, que se puede diferencial
    dentro  de la integral para obtener:
    $$
    \frac{\partial f}{\partial w}=\int_{u}^v
    \frac{\partial g}{\partial w}(x,w)dx.
    $$

    Finalmente, usa ésta fórmula y los incisos anteriores
    para demostrar la ecuación \eqref{Eqn:FormulaDerivadaInt}. 
  </ol>
  
  
</ol>

  
       \end{document}
