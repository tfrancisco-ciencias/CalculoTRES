
\documentclass{article}
\usepackage{amssymb, latexsym, amsmath, amsthm, amsfonts, amsbsy, enumerate, tabularx, graphicx}
\newtheorem{definicion}{Definici\'on}

\newtheorem{teorema}{Teorema}
\newtheorem{prop}{Proposici\'on}
\newtheorem{coro}{Corolario}
\newtheorem{lema}{Lema}
\theoremstyle{definition}
\newtheorem{nota}{Nota}
\newtheorem{notas}{Notas}
\newtheorem{ejemplo}{Ejemplo}
\newtheorem{ejemplos}{Ejemplos}
%\newtheorem{definicion}{Definici\'on}
\newtheorem{notacion}{Notaci\'on}
\newcommand{\sen}{\operatorname{sen}}
\newcommand{\senh}{\operatorname{senh}}
\newcommand{\csch}{\operatorname{csch}}
\newcommand{\sech}{\operatorname{sech}}
\newcommand{\arcsec}{\operatorname{arcsec}}
\newcommand{\arcsen}{\operatorname{arcsen}}
\newcommand{\id}{\operatorname{id}}
\begin{document}




  \section*{T4}
  
  
  
  \begin{enumerate}

	\item Bosqueja los siguientes campos vectoriales.

	
	\begin{enumerate}
	\item $F(x,y)=e^{x^2+y^2}(-y,x)$.
	\item $F(x,y)=(x+1,y-2)$.
	\item $F(x,y)=log(x^2+y^2)(x,y)$.
	\item $F(x,y)=(\cos(x),\sen(x))$.
	\end{enumerate}


  \item Calcula los siguientes l\'imites, si es que existen, si no prueba que no existen.
  
  \begin{enumerate}
  \item $\lim_{(x,y)\to (0,0)} \frac{(x+y)^2-(x-y)^2}{xy}$
  \item $\lim_{(x,y)\to (0,0)} \frac{\sen(xy)}{y}$
  \item $\lim_{(x,y)\to (0,0)} \frac{x^3-y^3}{x^2+y^2}$
  \end{enumerate}
  
  \item Calcula los siguientes l\'imites, si es que existen, si no prueba que no existen.


	\begin{enumerate}
	\item $\lim_{(x,y,z)\to (0,0,0)}\frac{\sen(xyz)}{xyz}$
	\item $\lim_{(x,y,z) \to (0,0,0)} \frac{x^2+3y^2}{x+1}$
	\item $\lim_{(x,y,z)\to (0,0,0)} \frac{2x^2y\cos(z)}{x^2+y^2}$
\end{enumerate}	  
  
  
  \item Calcula $\lim_{(x,y)\to (0,0)} (2x^2+2y^2)\log(x^2+y^2)$.
  
  Sugerencia: usa coordenadas polares.

  \item Demuestra que $\lim_{(x,y,z)\to (0,0,0)}\frac{xyz}{x^2+y^2+z^2}=0$.

\item Asi como el campo $F(x,y)=(-y,x)$ es perpendicular a las curvas $x^2+y^2=$constante, encuentra un campo vectorial $F(x,y)$, tal que es perpendicular a las curvas $2x^2+3y^2=$constante.


\item Sea $\{U_\alpha\}_{\alpha \in \Lambda}$ una colecci\'on de abiertos de $\mathbb{R}^n$. Demuestra que
$\cup_{\alpha \in \Lambda}U_\alpha$ es abierto de $\mathbb{R}^n$.

\item 
	\begin{enumerate}
	\item Fija $p_0\in \mathbb{R}^n$. Prueba que si $s<r$ entonces $B_s(p_0) \subseteq B_r(p_0)$.
	
	\item Sean $U, V$ abiertos de $\mathbb{R}^n$. Demuestra que $U\cap V$ es abierto.
	
	\end{enumerate}

	\item Considera el rect\'angulo $[a,b]\times [c,d]$ en $\mathbb{R}^2$. Encuentra sus puntos frontera.	
	
	\item Si $A $ es un subconjunto de $\mathbb{R}^n$, por $A^o$ denotamos a la uni\'on de todos los subconjuntos
	abiertos que est\'an contenidos en $A$ (si $A$ no contienen abiertos, $A^o=\emptyset$). 
	
	\begin{enumerate} 
	\item Prueba  que
	$A^o$ es un conjunto abierto. 

	\item Prueba $A^o \subseteq A$.
	\item $(A^o)^o=A^o$.
	\item $(A\cap B)^o=A^o \cap B^o$.
	\item Da un ejemplo que muestre que $(A\cup B)^o=A^o\cup B^o$ no siempre se da.
\end{enumerate}	

  \end{enumerate}
  


  
       \end{document}
