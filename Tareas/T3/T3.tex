
\documentclass{article}
\usepackage{amssymb, latexsym, amsmath, amsthm, amsfonts, amsbsy, enumerate, tabularx, graphicx}
\newtheorem{definicion}{Definición}

\newtheorem{teorema}{Teorema}
\newtheorem{prop}{Proposición}
\newtheorem{coro}{Corolario}
\newtheorem{lema}{Lema}
\theoremstyle{definition}
\newtheorem{nota}{Nota}
\newtheorem{notas}{Notas}
\newtheorem{ejemplo}{Ejemplo}
\newtheorem{ejemplos}{Ejemplos}
%\newtheorem{definicion}{Definición}
\newtheorem{notacion}{Notación}
\newcommand{\sen}{\operatorname{sen}}
\newcommand{\senh}{\operatorname{senh}}
\newcommand{\csch}{\operatorname{csch}}
\newcommand{\sech}{\operatorname{sech}}
\newcommand{\arcsec}{\operatorname{arcsec}}
\newcommand{\arcsen}{\operatorname{arcsen}}
\newcommand{\id}{\operatorname{id}}
\begin{document}




  \section*{T 3}


<ol>

<div class="ejercicio-box"> <h2 class="number-title"> Ejercicio</h2>\label{Ejer:BaseConProductoCruz} Sean $u,v\in \mathbb{R}^3$, dos vectores distintos de cero y linealmente
  independientes.

<ol>
<div class="ejercicio-box"> <h2 class="number-title"> Ejercicio</h2> Demuestra que el conjunto
  $\{u,v, u\times v\}$ es linealmente independiente.
  <div class="ejercicio-box"> <h2 class="number-title"> Ejercicio</h2> Es un hecho, no lo demuestres, que un conjunto linealmente independiente de tres vectores genera
  $\mathbb{R}^3$. Usando lo anterior muestra que, si $p$ es perpendicular a $u\times v$, entonces existen
  $s,t\in \mathbb{R}$ tales que $p=su+tv$ (es decir, $p$ está en el plano generado por $u$ y $v$).
  
  </ol>

<div class="ejercicio-box"> <h2 class="number-title"> Ejercicio</h2> Sean $u,v\in \mathbb{R}^3$ dos vectores unitarios y perpendiculares y sea $P:\mathbb{R}^3 \to \mathbb{R}^3$,
la proyección ortogonal al plano generado por $u$ y $v$, es decir
$$
P(p)= \langle u,p\rangle u + \langle v, p \rangle v.
$$

<ol>
<div class="ejercicio-box"> <h2 class="number-title"> Ejercicio</h2> Toma $w=u\times v$ y denota $P_w$ la proyección ortogonal a la recta generada por $w$. Demuestra que para todo punto $p\in \mathbb{R}^3$, $P(p)=p-P_{w}(p)$. 
	
	Sugerencia: empieza probando que $p-P_w(p)$ es ortogonal a $w$ por lo tanto, usando el ejercicio \ref{Ejer:BaseConProductoCruz}, $p-P_w(p)$ está en el plano generado por $u$ y $v$.


<div class="ejercicio-box"> <h2 class="number-title"> Ejercicio</h2> Concluye que,  para todo
$p\in \mathbb{R}^3$, se cumple
$$
p=\langle p, u \rangle u + \langle p, v \rangle v + \langle p, w \rangle w.
$$
</ol>

<div class="ejercicio-box"> <h2 class="number-title"> Ejercicio</h2> Para las siguientes 	funciones, describe los conjuntos de nivel para los valores dados.
	
	<ol>
%	<div class="ejercicio-box"> <h2 class="number-title"> Ejercicio</h2> $f(x,y)=x^2+y^2$, $c=-1,0,1.$
	<div class="ejercicio-box"> <h2 class="number-title"> Ejercicio</h2> $f(x,y)=2x^2+4y^2$, $c=-2,0,2$.
%	<div class="ejercicio-box"> <h2 class="number-title"> Ejercicio</h2> $f(x,y)=x^2-y^2$, $c=-3,0,3$.
	<div class="ejercicio-box"> <h2 class="number-title"> Ejercicio</h2> $f(x,y)=5x+7y$, $c=-5,0,7$.
%	<div class="ejercicio-box"> <h2 class="number-title"> Ejercicio</h2> $f(x,y)=(x+y)^2$, $c=-4,0,4$.
	<div class="ejercicio-box"> <h2 class="number-title"> Ejercicio</h2> $f(x,y)=(2x-y)^2$, $c=-1,0,2$.
%	<div class="ejercicio-box"> <h2 class="number-title"> Ejercicio</h2> $f(x,y)=y+\log(x^2)$, $c=-3,0,3$.
	<div class="ejercicio-box"> <h2 class="number-title"> Ejercicio</h2> $f(x,y)=y+3e^x$, $c=-2,0,2$.
%	<div class="ejercicio-box"> <h2 class="number-title"> Ejercicio</h2> $f(x,y)=\min\{|x|,|y| \}$, $c=0,1,2$.
</ol>		
	
	<div class="ejercicio-box"> <h2 class="number-title"> Ejercicio</h2> Para las siguientes funciones, describe los conjuntos de nivel para los valores dados.
	<ol>
	%<div class="ejercicio-box"> <h2 class="number-title"> Ejercicio</h2> $f(x,y,z)=x^2+y^2$, $c=-1,0,1$.
	<div class="ejercicio-box"> <h2 class="number-title"> Ejercicio</h2> $f(x,y,z)=x^2+y^2+z^2$, $c=-1,0,1$.
	%<div class="ejercicio-box"> <h2 class="number-title"> Ejercicio</h2> $f(x,y,z)=x^2+y^2-z^2$, $c=-2,0,2$.
	<div class="ejercicio-box"> <h2 class="number-title"> Ejercicio</h2> $f(x,y,z)=xyz$, $c=-4,0,4$.
	%<div class="ejercicio-box"> <h2 class="number-title"> Ejercicio</h2> $f(x,y,z)=\log(x)+\log(y)+\log(z)$, $c=-4,0,4.$
	<div class="ejercicio-box"> <h2 class="number-title"> Ejercicio</h2> $f(x,y,z)=e^{x^2}e^{y^3}e^{z^2}$, $c=-5,0,5$.
	%<div class="ejercicio-box"> <h2 class="number-title"> Ejercicio</h2> $f(x,y,z)=\max\{ |x|, |y|, |z| \}$, $c=-2,0,2$.
	<div class="ejercicio-box"> <h2 class="number-title"> Ejercicio</h2> $f(x,y,z)=2x-y+3z$, $c=-1,0,1$.
	</ol>
	
	
	<div class="ejercicio-box"> <h2 class="number-title"> Ejercicio</h2> Usa coordenadas polares para describir las curvas de nivel de la función dada por
	$$
	f(x,y)=\left\{
	\begin{array}{cc}
	\frac{2xy}{x^2+y^2} & (x,y)\ne (0,0)\\
	0 & (x,y)=(0,0).
	\end{array}
	\right.
	$$
	
	
	<div class="ejercicio-box"> <h2 class="number-title"> Ejercicio</h2> Construye una función $f(x,y)$, tal que el conjunto de nivel para $c=0$ consiste en una cantidad infinita
	de partes.
	
	<div class="ejercicio-box"> <h2 class="number-title"> Ejercicio</h2> Para cada una de las siguientes trayectorias, da un bosquejo de su traza (pueden usar matlab u otro software).
	<ol>
	%<div class="ejercicio-box"> <h2 class="number-title"> Ejercicio</h2> $\gamma(t)=(\cos(t), \sen(t))$, $t\in \mathbb{R}$.
	<div class="ejercicio-box"> <h2 class="number-title"> Ejercicio</h2> $\gamma(t)=(\cos(t), \sen(t), t)$, $t\in \mathbb{R}$.
	%<div class="ejercicio-box"> <h2 class="number-title"> Ejercicio</h2> $\gamma(t)=(t,t^2)$, $t\in \mathbb{R}$.
	<div class="ejercicio-box"> <h2 class="number-title"> Ejercicio</h2> $\gamma(t)=(\cos(t), \sen(t), e^t)$, $t\in \mathbb{R}$.
	%<div class="ejercicio-box"> <h2 class="number-title"> Ejercicio</h2> $\gamma(t)=2(\sen(\pi/4)\cos(t), \sen(\pi/4)\sen(t), \cos(\pi/4))$, $t\in \mathbb{R}$.
	</ol>
	


</ol>
  
       \end{document}
