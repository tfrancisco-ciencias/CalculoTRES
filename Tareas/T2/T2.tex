
\documentclass{article}
\usepackage{amssymb, latexsym, amsmath, amsthm, amsfonts, amsbsy, enumerate, tabularx, graphicx}
\newtheorem{definicion}{Definici\'on}

\newtheorem{teorema}{Teorema}
\newtheorem{prop}{Proposici\'on}
\newtheorem{coro}{Corolario}
\newtheorem{lema}{Lema}
\theoremstyle{definition}
\newtheorem{nota}{Nota}
\newtheorem{notas}{Notas}
\newtheorem{ejemplo}{Ejemplo}
\newtheorem{ejemplos}{Ejemplos}
%\newtheorem{definicion}{Definici\'on}
\newtheorem{notacion}{Notaci\'on}
\newcommand{\sen}{\operatorname{sen}}
\newcommand{\senh}{\operatorname{senh}}
\newcommand{\csch}{\operatorname{csch}}
\newcommand{\sech}{\operatorname{sech}}
\newcommand{\arcsec}{\operatorname{arcsec}}
\newcommand{\arcsen}{\operatorname{arcsen}}
\newcommand{\id}{\operatorname{id}}
\begin{document}




  \section*{T2}
  
  \begin{enumerate}
       \item Supon que $f:\mathbb{R}^2 \to \mathbb{R}$ es lineal y que $f(0,1)=3, f(1,1)=-1$. Calcula
       \begin{enumerate}
       \item $f(0,5)$,
       \item $f(2,2)$,
         \item $f(-1,0)$.
       \end{enumerate}

       \item Sea $\sigma$ una permutaci\'on del conjunto $\{1,2,3\}$. Define $F:\mathbb{R}^3\to \mathbb{R}^3$ por
  $F(p_1,p_2,p_3)=(p_{\sigma(1)}, p_{\sigma(2)}, p_{\sigma(3)})$. Prueba que $F$ es una funci\'on lineal.


       \item Todas las expresiones que siguen, excepto una, dan funciones lineales, por lo que se pueden escribir de la forma $Ap$, donde $A$ es una matriz y $p$ un vector. Encuentra la expresi\'on que no es lineal y para las otras  escribelas  de la forma $Ap$.

       
       \begin{enumerate}
       \item 
       $$
       \left[
       \begin{array}{cc}
       p_1+p_2 \\
       p_1-p_2
       \end{array}
       \right]	
       $$
       para $p\in \mathbb{R}^2$.
       \item $2p_1+p_3$, para  $p\in \mathbb{R}^3$.
       \item $2p_1+p_3$, para $p\in \mathbb{R}^4$.
       \item 
       $$
       \left[
       \begin{array}{c}
       p_1-2p_2 \\
       p_3+4p_4
       p_1+p_2+p_3+p_4
       \end{array}
       \right]
       $$
       para $p\in \mathbb{R}^4$.
       
	\item 
	$$
	\left[
	\begin{array}{c}
	0 \\
	p_1-2p_2\\
	p_3 \\
	p_3-p_1
	\end{array}
	\right]
	$$       
       con $p\in\mathbb{R}^3$.
       \end{enumerate}

	\item Sean $A,B$ dos matrices de $m \times n$. Demuestra que si $A\mathbf{p}=B\mathbf{p}$,
	para todo $\mathbf{p}\in \mathbb{R}^n$, entonces $A=B$. 
	
	Sugerencia: ve tomando $\mathbf{p}=\mathbf{e}_i$, $i=1,\dots, n$.



        
\item Considera $\theta\in [0, 2\pi)$ y la matriz 
$$
M=\left[
\begin{array}{cc}
\cos(\theta ) & -\sen(\theta) \\
\sen(\theta) & \cos(\theta)
\end{array}
\right]
$$

\begin{enumerate}
\item Tome el vector en el circulo unitario $u=(\cos(\varphi), \sen(\varphi))$. Demuestra que
$$
Mu=\left[
\begin{array}{c}
\cos(\theta+\varphi) \\
\sen(\theta+\varphi)
\end{array}
\right]
$$


Por lo tanto, la matriz $M$ representa la transformaci\'on que es rotar el plano un \'angulo $\theta$
(asi que las rotaciones son funciones lineales).

\item Dado cualquier vector $u\in \mathbb{R}^2$, usa coordenadas polares para escribir $u=(r\cos(\varphi), r\sen(\varphi))$, con $r\geq 0$ y $\varphi\in [0,2\pi)$. Demuestra que si $Mu=0$ entonces $u=0$.
\end{enumerate}


	
	\item Considera la funci\'on $F:\mathbb{R}^3\to  \mathbb{R}^3$ dada por
	$F(x,y,z)=(x-y+2z, y-2z, 4z)$. Resuelve el sistema de ecuaciones, para $x,y,z$
	$$
	\begin{array}{c}
	u=x-y+2z \\
	v=y-2z\\
	w=4z
	\end{array}
	$$
	para encontrar la funci\'on inversa de $F$.


        \item Verifica que 
	$$
	\left[
	\begin{array}{cc}
	a & b \\
	c & d
	\end{array}
	\right] 
	\left[
	\begin{array}{cc}
	d & -b \\
	-c & a
	\end{array}
	\right]=
	\left[
	\begin{array}{cc}
	ad-bc & 0 \\
	0 & ad-bc 
	\end{array}
	\right]
	$$

	Concluye que una matriz de $2 \times 2$ es invertible sii su determinante es distinto de cero y
	$$
\left[
	\begin{array}{cc}
	a & b \\
	c & d
	\end{array}
	\right]^{-1} 
	=\frac{1}{ad-bc}\left[
	\begin{array}{cc}
	d & -b \\
	-c & a
	\end{array}
	\right]
	$$
	
	
	\item Con ayuda del ejercicio encuentra las inversas de las siguientes funciones.
	\begin{enumerate}
	\item $F(x,y)=(2x-y, x+5y)$
	\item $F(x,y)=(x+y, x-y)$
	\item $F(x,y)=(5x+7y, y)$
	\end{enumerate}			
	
	
     \item Sean $u,v, w \in \mathbb{R}^2$ tres
       puntos en el circulo unitario tal que lo dividen en tres arcos de circunferencia cada uno de la misma longitud.
       \begin{enumerate}
       \item Prueba que la transformaci\'on que rota a el plano un \'angulo $2\pi/3$ (en cualquier direcci\'on)
         deja al vector $u+v+w$ fijo. De lo anterior concluye que $u+v+w=0$.
       \item Usando el inciso anterior prueba que, para todo \'angulo $\theta$
         \begin{eqnarray*}
         \sen(\theta)+\sen(\theta+2\pi/3)+\sen(\theta+4\pi/3)&=&0, \\
         \cos(\theta)+\cos(\theta+2\pi/3)+\cos(\theta+4\pi/3)&=&0 .
         \end{eqnarray*}

       \item Generaliza el inciso anterior para demostrar
         \begin{eqnarray*}
           \sum_{k=1}^n \cos(\theta+ 2\pi k/n)=0, \quad  \sum_{k=1}^n \sen(\theta +2\pi k /n)=0.
         \end{eqnarray*}
       \end{enumerate}



  \end{enumerate}

  
       \end{document}
