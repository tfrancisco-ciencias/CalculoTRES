\documentclass[12pt]{article}
\setlength{\topmargin}{-.75in}\addtolength{\textheight}{2.00in}
\setlength{\oddsidemargin}{.00in}\addtolength{\textwidth}{.75in}
\usepackage{amsmath,amssymb}
\nofiles

\pagestyle{empty}
\setlength{\parindent}{0in}
\newcommand{\sen}{\operatorname{sen}}


\begin{document}

\noindent{\sc {\bf{\large Quiz 15 }}
           \hfill C\'alculo 3, semestre 2020-2}
\bigskip

\noindent {\sc{
            { \Large Nombre: \underline {\hspace {10 cm }}}}}
            
\bigskip
\bigskip
\bigskip






\begin{enumerate}

\item (2pts) Mostrar que $x^3z^2-z^3yx=0$ es soluble para $z$, en funci\'on
  de $(x,y)$, cerca de $(1,1,1)$. Adem\'as, calcular $\partial_xz(1,1)$
  y $\partial_yz(1,1)$.

\vspace{5cm}


\item (3 pts) Considera $f(x,y,z)=z^4+2yz+x$ y la superficie de nivel:
$$
S=\{(x,y,z): f(x,y,z)=0\}.
$$

Muestra que $(0,2,0)$ est\'a en la superficie de nivel y que existe
una funci\'on $g(x,y)$, definida cerca de $(0,2)$, tal que $f(x,y,g(x,y))=0$.


\end{enumerate}
 





  
\end{document}
