\documentclass[12pt]{article}
\setlength{\topmargin}{-.75in}\addtolength{\textheight}{2.00in}
\setlength{\oddsidemargin}{.00in}\addtolength{\textwidth}{.75in}
\usepackage{amsmath,amssymb}
\nofiles

\pagestyle{empty}
\setlength{\parindent}{0in}



\begin{document}

\noindent{\sc {\bf{\large Quiz 3 }}
           \hfill C\'alculo 3, semestre 2020-2}
\bigskip

\noindent {\sc{
            { \Large Nombre: \underline {\hspace {10 cm }}}}}
            
\bigskip
\bigskip
\bigskip


\begin{enumerate}

\item (2 pts) Sea $M\subset \mathbb{R}^3$ un plano. Da la definici\'on anal\'itica (usando distancias) de la proyeccion
  ortogonal de un punto $p\in \mathbb{R}^3$ al plano $M$. 
\vspace{3cm}
  
\item (3 pts) Describe las curvas de nivel de la funci\'on $f:\mathbb{R}^2\to \mathbb{R}$ dada por $f(x)=x^2-y^2$.



  \end{enumerate}


  \vspace{3cm}

\noindent{\sc {\bf{\large Quiz 3 }}
           \hfill C\'alculo 3, semestre 2020-2}
\bigskip

\noindent {\sc{
            { \Large Nombre: \underline {\hspace {10 cm }}}}}
            
\bigskip
\bigskip
\bigskip


\begin{enumerate}

\item (2 pts) Sea $M\subset \mathbb{R}^3$ un plano. Da la definici\'on anal\'itica (usando distancias) de la proyeccion
  ortogonal de un punto $p\in \mathbb{R}^3$ al plano $M$. 
\vspace{3cm}
  
\item (3 pts) Describe las curvas de nivel de la funci\'on $f:\mathbb{R}^2\to \mathbb{R}$ dada por $f(x)=x^2-y^2$.



  \end{enumerate}

  
  
\end{document}
