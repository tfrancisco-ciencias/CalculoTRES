
\documentclass{article}
\usepackage{amssymb, latexsym, amsmath, amsthm, amsfonts, amsbsy, enumerate, tabularx, graphicx}
\newtheorem{definicion}{Definición}

\newtheorem{teorema}{Teorema}
\newtheorem{prop}{Proposición}
\newtheorem{coro}{Corolario}
\newtheorem{lema}{Lema}
\theoremstyle{definition}
\newtheorem{nota}{Nota}
\newtheorem{notas}{Notas}
\newtheorem{ejemplo}{Ejemplo}
\newtheorem{ejemplos}{Ejemplos}
%\newtheorem{definicion}{Definición}
\newtheorem{notacion}{Notación}
\newcommand{\sen}{\operatorname{sen}}
\newcommand{\senh}{\operatorname{senh}}
\newcommand{\csch}{\operatorname{csch}}
\newcommand{\sech}{\operatorname{sech}}
\newcommand{\arcsec}{\operatorname{arcsec}}
\newcommand{\arcsen}{\operatorname{arcsen}}
\newcommand{\id}{\operatorname{id}}
\begin{document}




  

	<div class="definicion-box"><h2 class="number-title">  Definición</h2>
	Una función $f: \mathbb{R}^n \to \mathbb{R}$ se llama convexa si 
	$$f( tp+(1-t)q ) \leq  tf(p)+(1-t)f(q),$$ 
	para todos $p,q \in \mathbb{R}^n$ y todo $t \in [0,1]$. 
	
	Por ejemplo, en cálculo uno-dimensional, se prueba que si $f: I \to \mathbb{R}$ (donde $I$ es un intervalo)
	es diferenciable y $f''(x)>0$ para todo $x\in I$, entonces $f$ es convexa.
	
	Una curvas $C\subset \mathbb{R}^n$ se llama convexa si, para todos $p,q\in C$, el segmento que une
	$p$ con $q$ queda por arriba del segmento de  $C$ que va de $p$ a $q$.
	
	</div>


        \section*{T5}
  
  <ol>

<div class="ejercicio-box"> <h2 class="number-title"> Ejercicio</h2> Ecuación del plano, forma normal. 

Sean $u\ne 0, v\ne 0$, dos vectores en $\mathbb{R}^3$, linealmente independientes. Considera la ecuación 
paramétrica
$$
p=p_0+tu+sv
$$	
con $p_0=(x_0,y_0,z_0)$. Sea $(A,B,C)=u\times v$. Demuestra que todo punto $p=(x,y,z)$ en el plano
generado por $u$ y $v$ satisface
$$
A(x-x_0)+B(y-y_0)+C(z-z_0)=0.
$$

Al vector $(A,B,C)$ se le llama un  vector normal al plano.


<div class="ejercicio-box"> <h2 class="number-title"> Ejercicio</h2> Sean $u,v$ y $w$, vectores ortogonales y con norma 1. Demuestra que si
  $$
  p=\alpha u + \beta v + \gamma w
  $$
  entonces $\alpha = \langle p, u \rangle$, $\beta=\langle p, v \rangle$ y $\gamma= \langle p, w \rangle$.
  
    <div class="ejercicio-box"> <h2 class="number-title"> Ejercicio</h2> Sean $p,q\in \mathbb{R}^n$. Demuestra que los vectores $\|p\|q+\|q\|p$ y $\|p\|q-\|q\|p$ son ortogonales.
 

      <div class="ejercicio-box"> <h2 class="number-title"> Ejercicio</h2> Sean $u,v\in \mathbb{R}^3$. Demuestra que : $\|u\times v \|^2+ (\langle u,v \rangle)^2=\|u\|^2\|v\|^2$.

    <div class="ejercicio-box"> <h2 class="number-title"> Ejercicio</h2> Dados dos vectores no cero, $p,q\in \mathbb{R}^n$, prueba que el vector $v=\|p\|q+\|q\|p$ bisecta a el
    ángulo entre $p$ y $q$.
    
	
	<div class="ejercicio-box"> <h2 class="number-title"> Ejercicio</h2> Una función de producción tipo Cobb-Douglas, en dos dimensiones, es una función
	de la forma $f(x,y)=x^\alpha y^\beta$, donde $x,y \geq 0$ y $0<\alpha, \beta$.
	
	Para $c>0$ considera la curva de nivel
	$$
	C=\{ (x,y)\in \mathbb{R}^2: x, y > 0, x^\alpha y^\beta =c\}.
	$$
	
	
	
	
	Este ejercicio demuestra que $C$ es una curva convexa. 
	<ol>
	<div class="ejercicio-box"> <h2 class="number-title"> Ejercicio</h2> Demuestra que, para todo $(x,y)\in C$, $y=\frac{c^{1/\beta}}{x^{\alpha/\beta}}$.
	<div class="ejercicio-box"> <h2 class="number-title"> Ejercicio</h2> Demuestra que la función $g(s)=\frac{1}{s^{\alpha / \beta}}$, $s >0$, es convexa. 
	<div class="ejercicio-box"> <h2 class="number-title"> Ejercicio</h2> Demuestra  que $C$ es una curva convexa.
	
	Sugerencia: Toma $(x_1,y_1), (x_2,y_2)\in C$ y sin pérdida de generalidad podemos suponer $x_1<x_2$. Toma
	$(x,y)\in C$ un punto entre $(x_1,y_1)$ y $(x_2,y_2)$ (por lo tanto $x_1<x<x_2$). Si $t\in [0,1]$
	es tal que $x=(1-t)x_1+tx_2$, usando $y=g(x)$ y el ejercicio anterior, demuestra que  $y\leq (1-t)y_1+ty_2$.
	
	</ol>



      <div class="ejercicio-box"> <h2 class="number-title"> Ejercicio</h2> Para los siguientes límites, calcula si existe y si no prueba que no existe.
        <ol>
        <div class="ejercicio-box"> <h2 class="number-title"> Ejercicio</h2> $\lim_{(x,y,z)\to (0,0,0)} \frac{2x^2y\cos(z)}{x^2+y^2}$
          <div class="ejercicio-box"> <h2 class="number-title"> Ejercicio</h2> $\lim_{(x,y)\to (0,0)} \frac{\cos(x)-1-(x^2/2)}{x^4+y^4}$
        </ol>
    
  <div class="ejercicio-box"> <h2 class="number-title"> Ejercicio</h2> Sea $F:\mathbb{R}^n \to \mathbb{R}^m$ una función lineal tal que
    $$
    \lim_{p\to 0} \frac{\|F(p)\|}{\|p\|}=0.
    $$

    Demuestra que $F$ es la función constante cero.

  <div class="ejercicio-box"> <h2 class="number-title"> Ejercicio</h2> Prueba que los siguientes conjuntos son abiertos:
    <ol>
    <div class="ejercicio-box"> <h2 class="number-title"> Ejercicio</h2> $A=\{(x,y): -1< x <1, -1<y <1  \}$.
  <div class="ejercicio-box"> <h2 class="number-title"> Ejercicio</h2> $B=\{(x,y): x< 0  \}$.
    <div class="ejercicio-box"> <h2 class="number-title"> Ejercicio</h2> $C=\{ (x,y): |x|>2  \}$.
    </ol>
    


  </ol>
  
  
       \end{document}
