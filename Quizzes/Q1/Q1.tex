\documentclass[12pt]{article}
\setlength{\topmargin}{-.75in}\addtolength{\textheight}{2.00in}
\setlength{\oddsidemargin}{.00in}\addtolength{\textwidth}{.75in}
\usepackage{amsmath,amssymb}
\nofiles

\pagestyle{empty}
\setlength{\parindent}{0in}



\begin{document}

\noindent{\sc {\bf{\large Quiz 1 }}
           \hfill Cálculo 3, semestre 2020-2}
\bigskip

\noindent {\sc{
            { \Large Nombre: \underline {\hspace {10 cm }}}}}
            
\bigskip
\bigskip
\bigskip


<ol>

<div class="ejercicio-box"> <h2 class="number-title"> Ejercicio</h2> (2 pts) Calcula el siguiente producto cruz:
  $$(3\hat{i}+2\hat{j}-5\hat{k})\times (-2\hat{i}+\hat{j}+2\hat{k})$$

\vspace{3.5cm}
  
  <div class="ejercicio-box"> <h2 class="number-title"> Ejercicio</h2> (3 pts) Encuentra el área del triángulo con vértices $(0,0), (5,2), (6,4)$.

  </ol>


  \vspace{3cm}

  
\noindent{\sc {\bf{\large Quiz 1 }}
           \hfill Cálculo 3, semestre 2020-2}
\bigskip

\noindent {\sc{
            { \Large Nombre: \underline {\hspace {10 cm }}}}}
            
\bigskip
\bigskip
\bigskip


<ol>

<div class="ejercicio-box"> <h2 class="number-title"> Ejercicio</h2> (2 pts) Calcula el siguiente producto cruz:
  $$(3\hat{i}+2\hat{j}-5\hat{k})\times (-2\hat{i}+\hat{j}+2\hat{k})$$

\vspace{3.5cm}
  
  <div class="ejercicio-box"> <h2 class="number-title"> Ejercicio</h2> (3 pts) Encuentra el área del triángulo con vértices $(0,0), (5,2), (6,4)$.

</ol>
  
\end{document}
