\documentclass[12pt]{article}
\setlength{\topmargin}{-.75in}\addtolength{\textheight}{2.00in}
\setlength{\oddsidemargin}{.00in}\addtolength{\textwidth}{.75in}
\usepackage{amsmath,amssymb}
\nofiles

\pagestyle{empty}
\setlength{\parindent}{0in}
\newcommand{\sen}{\operatorname{sen}}


\begin{document}

\noindent{\sc {\bf{\large Quiz 6 }}
           \hfill Cálculo 3, semestre 2020-2}
\bigskip

\noindent {\sc{
            { \Large Nombre: \underline {\hspace {10 cm }}}}}
            
\bigskip
\bigskip
\bigskip


<ol>

<div class="ejercicio-box"> <h2 class="number-title"> Ejercicio</h2> (2 pts) Sea $U$ un abierto en $\mathbb{R}^2$, $p_0=(x_0,y_0)\in U$
  y $f:U\to \mathbb{R}$ una
  función. Escribe la definición de que $f$ sea diferenciable en $p_0$.

  Nota  que $f$ es una función de DOS variables.
  
  
\vspace{5cm}  
  
<div class="ejercicio-box"> <h2 class="number-title"> Ejercicio</h2> (3 pts) Sea $F:\mathbb{R}^2 \to \mathbb{R}^3$ dada por $F(x,y)=(\sen(xy), \cos(xy),\tan(xy))$.
  Encuentra la matriz de derivadas parciales.
  </ol>


  
\end{document}
