
\documentclass{article}
\usepackage{amssymb, latexsym, amsmath, amsthm, amsfonts, amsbsy, enumerate, tabularx, graphicx}
\newtheorem{definicion}{Definición}

\newtheorem{teorema}{Teorema}
\newtheorem{prop}{Proposición}
\newtheorem{coro}{Corolario}
\newtheorem{lema}{Lema}
\theoremstyle{definition}
\newtheorem{nota}{Nota}
\newtheorem{notas}{Notas}
\newtheorem{ejemplo}{Ejemplo}
\newtheorem{ejemplos}{Ejemplos}
%\newtheorem{definicion}{Definición}
\newtheorem{notacion}{Notación}
\newcommand{\sen}{\operatorname{sen}}
\newcommand{\senh}{\operatorname{senh}}
\newcommand{\csch}{\operatorname{csch}}
\newcommand{\sech}{\operatorname{sech}}
\newcommand{\arcsec}{\operatorname{arcsec}}
\newcommand{\arcsen}{\operatorname{arcsen}}
\newcommand{\id}{\operatorname{id}}
\begin{document}




  \section*{T12 }
  <ol>
  

    
  <div class="ejercicio-box"> <h2 class="number-title"> Ejercicio</h2> Considera las funciones $F:\mathbb{R}^3\to \mathbb{R}^2$ y $G:\mathbb{R}^3\to \mathbb{R}^3$ dadas
    por
    $$
    F(x,y,z)=(x^2+y+z, 2x+y+z^2), \quad G(u,v,w)=(2uv^2w^2,w^2\sen(v),u^2e^v)
    $$
    <ol>
    <div class="ejercicio-box"> <h2 class="number-title"> Ejercicio</h2> Encuentra la matriz de derivadas parciales $D_{(x,y,z)}F$ y $D_{(u,v,w)}G$.
    <div class="ejercicio-box"> <h2 class="number-title"> Ejercicio</h2> Define $H=F\circ G$. Usa la regla de la cadena para calcular la matriz derivada
      parciales $D_{(u,0,w)}H$.
    </ol>


    
  <div class="ejercicio-box"> <h2 class="number-title"> Ejercicio</h2> Hallar la ecuación del plano tangente a las superficies
    dadas en los puntos indicados.
    <ol>
    <div class="ejercicio-box"> <h2 class="number-title"> Ejercicio</h2> $x^2+2y^2+3xz=10$ en $(1,2,1/3)$,
    <div class="ejercicio-box"> <h2 class="number-title"> Ejercicio</h2> $y^2-x^2=3$ en $(1,2,8)$,
    <div class="ejercicio-box"> <h2 class="number-title"> Ejercicio</h2> $xyz=1$ en $(1,1,1)$.
    </ol>


  %<div class="ejercicio-box"> <h2 class="number-title"> Ejercicio</h2> Dada una superficie $S$ en $\mathbb{R}^3$ y un punto $p_0$ en la superficie, vamos a
   % denotar $N,U,L$, a tres vectores unitarios con las características de que: $N$
   % es el vector normal al plano tangente a $S$ que pasa por $p_0$; $U$ es vector que indica
   % la dirección donde la razón de crecimiento, en $p_0$, es máxima; $L$ es el
   % vector donde la razón de crecimiento, en $p_0$, es cero.

    %Nota:  en general hay dos elecciones para dichos vectores, pues podemos tomar su negativo
    %(por ejemplo $-N$ o $N$).

    %Para las siguientes superficies calcula $N,U$ y $L$ para un punto general de la superficie.
    %<ol>
    %<div class="ejercicio-box"> <h2 class="number-title"> Ejercicio</h2> $S$ es la gráfica de la función $g(x,y)=1-x-y$ (un plano).
    %<div class="ejercicio-box"> <h2 class="number-title"> Ejercicio</h2> $S$ es la superficie dada por la ecuación $x^2+y^2-z^2=0$ (un cono). En este ejemplo
    %  se evita el vértica, pues el plano tangente en ese punto no está bien definido.
    %</ol>


  <div class="ejercicio-box"> <h2 class="number-title"> Ejercicio</h2> Recuerda que, para una superficie de nivel $S \subset \mathbb{R}^3$, de la función $g(x,y,z)$,
    la ecuación del plano tangente es
    $$
    \langle \nabla_{(x_0,y_0,z_0)}g, (x-x_0,y-y_0,z-z_0) \rangle =0
    $$
    donde $(x_0,y_0z_0)$ es un punto en la superficie $S$.

    Demuestra que, como caso especial, la fórmula del plano tangente a la gráfica de la función
    $f(x,y)$, se puede obtener de la ecuación anterior si se considera a la gráfica
    como una superficie de nivel de $F(x,y,z)=f(x,y)-z$.
 
    
  <div class="ejercicio-box"> <h2 class="number-title"> Ejercicio</h2> Considera la función $f(x,y)=-(1-x^2-y^2)^{1/2}$, definida para los puntos $(x,y)$
    con $x^2+y^2<1$. Prueba que el plano tangente a la gráfica de $f$, en el punto $(x_0,y_0,f(x_0,y_0))$
    es ortogonal al vector $(x_0,y_0,f(x_0,y_0))$.

    <div class="ejercicio-box"> <h2 class="number-title"> Ejercicio</h2> Usa la regla de la cadena para probar que, si $g:U\to \mathbb{R}$ es de clase
      $C^1$ en $U$, con $U$ un abierto de $\mathbb{R}^n$, entonces
      $$
      \nabla_{p}(1/g)=-\frac{1}{g(p)^2}\nabla_p g
      $$    
      donde suponemos que $g(p)\ne 0$, para toda $p\in U$.
    
  <div class="ejercicio-box"> <h2 class="number-title"> Ejercicio</h2> Sean $G:\mathbb{R}^m \to \mathbb{R}^n$, una función de clase $C^1$ en $\mathbb{R}^m$, con funciones
    coordenadas $G(q)=(g_1(q),\dots, g_n(q))$ y sea $f:\mathbb{R}^n \to \mathbb{R}$ una función
    de clase $C^1$ en $\mathbb{R}^n$ y sea $h=f\circ G$. Usa la regla de la cadena para demostrar que
    el gradiente de $h$ es una combinación lineal de los gradientes de las $g_k$, en específico:
    $$
    \nabla_{q_0} h= \sum_{k=1}^n \partial_{p_k}f(G(q_0)) \nabla_{q_0}g_k
    $$
    nota que $\partial_{p_k}f(g(q_0))$ es escalar y $\nabla_{q_0}g_k$ es vector.


    <div class="ejercicio-box"> <h2 class="number-title"> Ejercicio</h2> Encuentra el conjunto de puntos $(a,b,c)$ en $\mathbb{R}^3$, para los cuales las
      dos esferas: $(x-a)^2+(y-b)^2+(z-c)^2=1$ y $x^2+y^2+z^2=1$, se intersectan ortogonalmente.

      Nota: Dos superficies se intersectan ortogonalmente si, para todo punto en su intersección,
      los planos tangentes son ortogonales. 

      Sugerencia: ve las esferas como superficies de nivel y utiliza gradientes. 

  <div class="ejercicio-box"> <h2 class="number-title"> Ejercicio</h2>
    <ol>
    <div class="ejercicio-box"> <h2 class="number-title"> Ejercicio</h2> Considera la función $I:\mathbb{R}^3\to \mathbb{R}^3$ dada por $I(x,y,z)=(x,y,z)$.
      Demuestra que
      $$
      D_{(x,y,z)}I=\left[
        \begin{array}{ccc}
          1 & 0 & 0 \\
          0 & 1 & 0 \\
          0 & 0 & 1
        \end{array}
      \right]
      $$
    <div class="ejercicio-box"> <h2 class="number-title"> Ejercicio</h2> Encuentra todas las funciones diferenciables en $\mathbb{R}^3$,
      $F:\mathbb{R}^3 \to \mathbb{R}^3$, para las cuales

      $$
      D_{(x,y,z)}F=\left[
        \begin{array}{ccc}
          x & 0 & 0 \\
          0 & y & 0 \\
          0 & 0 & x
        \end{array}
      \right]
      $$

    <div class="ejercicio-box"> <h2 class="number-title"> Ejercicio</h2> Sean $p,q,r:\mathbb{R}\to \mathbb{R}$ funciones continuas en todo $\mathbb{R}$. Encuentra
      todas la funciones diferenciables en $\mathbb{R}^3$, $G:\mathbb{R}^3 \to \mathbb{R}^3$, para
      las cuales

      $$
      D_{(x,y,z)}G=\left[
        \begin{array}{ccc}
          p(x) & 0 & 0 \\
          0 & q(y) & 0 \\
          0 & 0 & r(z)
        \end{array}
      \right]
      $$

      
    </ol>

    </ol>

       \end{document}


