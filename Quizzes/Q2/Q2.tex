\documentclass[12pt]{article}
\setlength{\topmargin}{-.75in}\addtolength{\textheight}{2.00in}
\setlength{\oddsidemargin}{.00in}\addtolength{\textwidth}{.75in}
\usepackage{amsmath,amssymb}
\nofiles

\pagestyle{empty}
\setlength{\parindent}{0in}



\begin{document}

\noindent{\sc {\bf{\large Quiz 2 }}
           \hfill Cálculo 3, semestre 2020-2}
\bigskip

\noindent {\sc{
            { \Large Nombre: \underline {\hspace {10 cm }}}}}
            
\bigskip
\bigskip
\bigskip


<ol>

\item (1 pts) Supon que $f:\mathbb{R}^2 \to \mathbb{R}$ es lineal y que $f(1,0)=2, f(-1,-1)=1$. Calcula el 
 valor de $f(0,2)$.
\vspace{4cm}
  
  \item (2 pts) Considera la función lineal $F:\mathbb{R}^2 \to \mathbb{R}^3$ dada por
  $$F(x,y)=(x+y, x-y, 5x+2y)$$ 
  
  Encuentra la matriz $A$, tal que  $F(p)=Ap$, para todo $p\in \mathbb{R}^2$.


\vspace*{4cm}

\item (2 pts)  Sean $F:\mathbb{R}^n \to \mathbb{R}^m$ y $G:\mathbb{R}^m \to \mathbb{R}^k$ dos funciones lineales,
demuestra que la composición $G\circ F:\mathbb{R}^n \to \mathbb{R}^k$,  también es una función  lineal.

  </ol>


  \vspace{3cm}
  
\end{document}
