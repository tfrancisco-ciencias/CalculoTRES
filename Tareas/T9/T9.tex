
\documentclass{article}
\usepackage{amssymb, latexsym, amsmath, amsthm, amsfonts, amsbsy, enumerate, tabularx, graphicx}
\newtheorem{definicion}{Definición}

\newtheorem{teorema}{Teorema}
\newtheorem{prop}{Proposición}
\newtheorem{coro}{Corolario}
\newtheorem{lema}{Lema}
\theoremstyle{definition}
\newtheorem{nota}{Nota}
\newtheorem{notas}{Notas}
\newtheorem{ejemplo}{Ejemplo}
\newtheorem{ejemplos}{Ejemplos}
%\newtheorem{definicion}{Definición}
\newtheorem{notacion}{Notación}
\newcommand{\sen}{\operatorname{sen}}
\newcommand{\senh}{\operatorname{senh}}
\newcommand{\csch}{\operatorname{csch}}
\newcommand{\sech}{\operatorname{sech}}
\newcommand{\arcsec}{\operatorname{arcsec}}
\newcommand{\arcsen}{\operatorname{arcsen}}
\newcommand{\id}{\operatorname{id}}
\begin{document}




  \section*{T9 }


  <ol>



  \item\label{Ejer:DesigualdadesAplicandoTVM}
    Utiliza el Teorema de Valor Medio para probar las siguientes desigualdades:
<ol>
\item $|\sen(a)-\sen(b)| \leq |a-b|$,
\item $|\sen(2x)-\sen(x)| \leq |x|$.
</ol>


\item Sean $f,g:[a,\infty)\to \mathbb{R}$, continuas en $[a,\infty)$ y diferenciables en $(a,\infty)$ con
$f(a) \leq g(a)$.

<ol>
\item Si $f'(x) < g'(x)$, para todo $x>a$, usa el T.V.M. para demostrar que $f(x) < g(x)$, para toda $x>a$.
\item Usar el inciso anterior para probar que $\sqrt{1+x}< 1+\frac{x}{2}$ para todo $x>0$.
</ol>    
  
\item Sea $f:B_r(x_0,y_0)\subset \mathbb{R}^2 \to \mathbb{R}$ una función diferenciable en todo
  punto $(x,y)$, con $(x,y)\in B_r(x_0,y_0)$. Demuestra que, para todos $x_1,x_2$ con $|x_1-x_0|<r$,
  $|x_2-x_0|<r$, se tiene que existe $c$ entre $x_1$ y $x_2$ tal que
  $$
  |f(x_1,y_0)-f(x_2,y_0)| = |\partial_xf(c,y_0)||x_1-x_2|.
  $$

  \item Para las siguientes funciones, usa el criterio
    de las derivadas parciales para encontrar  un dominio
    para el cual la función sea diferencible en
    todo punto del dominio.
    <ol>
    \item $f(x,y)=e^{\sqrt{x^2+y^2}}$,  $(x,y)\in \mathbb{R}^2$,
    \item $f(x,y)=\log(e^{\sqrt[4]{(x-1)^2+(y-2)^2}}+e^{\sqrt{(x-5)^2+(y-6)^2}})$,
      $(x,y)\in \mathbb{R}^2$,
      \item $f(x,y)=\log(1+\sqrt{(x-1)^2+(y+1)^2})$, $(x,y)\in \mathbb{R}^2$.
    </ol>
    


      \item Usa el teorema de la igualdad de derivadas parciales mixtas
    para dos variables para probar que si $f:\mathbb{R}^3\to \mathbb{R}$
    es de clase $C^3$, entonces
    $$
    \partial^{3}_{xyz}f=\partial_{yzx}^3 f
    $$


    
    \item Para cada una de las siguientes funciones
    calcula $\partial_{x}\partial_yf(x,y)$ y $\partial_y\partial_xf(x,y)$.
    ¿ Qué notas ?
    <ol>
    \item $f(x,y)=x^2+xy^2+y^3$
    \item $f(x,y)=\log(x^2+y^4)$
    \item $f(x,y)=e^{3x+2xy+y^2}$                 
    </ol>

    
  \item Considera  la función $g(x,t)=2+e^{-t}\sen(x)$,
    $(t,x)\in \mathbb{R}^2$.

    <ol>
    \item Demuestra que $g$ satiface la
      ecuación del calor:
      $$
      \partial_t g= \partial_x^2g.
      $$
      Aquí $g(x,t)$, representa la temperatura de una varilla de metal
      en la posición $x$ al tiempo $t$.
    \item Esbozar la gráfica de $g$, para $t\geq 0$.
    \item ¿ Qué sucede con $g(x,t)$ cuando $t\to \infty$ ?
    Interpreta éste límite en términos del comportamiento del calor
    en la varilla.

  </ol>
  

    <div class="definicion-box"><h2 class="number-title">  Definición</h2>
      Una función $f:U\to \mathbb{R}$, definida en un abierto,
      se llama armónica en $U$ si las
      derivadas parciales de segundo orden $\partial_x^2 f$ y $\partial_y^2 f$,
      existen en todo punto de $U$, son continuas en todo $U$ y
       $$
       \partial_x^2f(x,y)+\partial_y^2f(x,y)=0
       $$ 
       para todo $(x,y)\in U$. 
     </div>

  \item  Para las siguientes funciones,
    determina cuales son funciones armónicas. 

       
    <ol>
    \item $f(x,y)=x^2-y^2$, $(x,y)\in \mathbb{R}^2$;
    \item $f(x,y)=e^y\cos(x)$, $(x,y)\in \mathbb{R}^2$; 
    \item $f(x,y)=e^y\sen(x)$, $(x,y)\in \mathbb{R}^2$;
    \item $f(x,y)=\log(x^2+y^2)$,
      $(x,y)\in \mathbb{R}^2$, $(x,y)\ne (0.0)$;
    \item $f(x,y)=x^3-3x^2y-3yx^3+y^3$, $(x,y)\in \mathbb{R}^2$.
    </ol>		       

  \item Demuestra que toda función
    lineal $f:\mathbb{R}^2\to \mathbb{R}$ es armónica.
    
                
  \item Considera la función $f(x,y)=ax^2+by^2+cxy$,
    donde $a,b,c\in \mathbb{R}$ son parámetros. Encuentra
    los valores de $a$, $b$ y $c$ para los cuales
    <ol>
    \item $\partial_x^2 f(x,y)+\partial_y^2f(x,y) >0$, para todo $(x,y)$,
    \item $\partial_x^2f(x,y)+\partial_y^2f(x,y)=0$, para todo $(x,y)$,
    \item $\partial_x^2f(x,y)+\partial_y^2f(x,y)<0$, para todo $(x,y)$.
    </ol>
                  
 
    
  \item (Vogel) Este ejercicio da un ejemplo
    donde las parciales mixtas no son iguales.

    Considera la función
    $$
    f(x,y)=\left\{
      \begin{array}{cc}
        \frac{xy^3-x^3y}{x^2+y^2} & (x,y)\ne (0,0)\\
        0 & (x,y)=(0,0)
      \end{array}
    \right.
    $$
    Prueba
    <ol>
    \item
      $$
      \partial_xf(x,y)=\left\{
        \begin{array}{cc}
          \frac{y^3-3x^2y}{x^2+y^2}- \frac{2x(xy^3-x^3y)}{(x^2+y^2)^2}
          & (x,y)\ne (0,0)\\
          0 & (x,y)=(0,0)
        \end{array}
      \right.
      $$
    \item
      $$
      \partial_y f(x,y)=\left\{
        \begin{array}{cc}
          \frac{3xy^2-x^3}{x^2+y^2}- \frac{2y(xy^3-x^3y)}{(x^2+y^2)^2}
          & (x,y)\ne (0,0)\\
          0 & (x,y)=(0,0)
        \end{array}
      \right.
      $$
      
    \item $\partial_x \partial_yf(0,0)=-1$ y $\partial_y\partial_xf(0,0)=1$
    </ol>



    
  </ol>
  
  
       \end{document}
