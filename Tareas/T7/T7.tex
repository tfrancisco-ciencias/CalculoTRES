
\documentclass{article}
\usepackage{amssymb, latexsym, amsmath, amsthm, amsfonts, amsbsy, enumerate, tabularx, graphicx}
\newtheorem{definicion}{Definición}

\newtheorem{teorema}{Teorema}
\newtheorem{prop}{Proposición}
\newtheorem{coro}{Corolario}
\newtheorem{lema}{Lema}
\theoremstyle{definition}
\newtheorem{nota}{Nota}
\newtheorem{notas}{Notas}
\newtheorem{ejemplo}{Ejemplo}
\newtheorem{ejemplos}{Ejemplos}
%\newtheorem{definicion}{Definición}
\newtheorem{notacion}{Notación}
\newcommand{\sen}{\operatorname{sen}}
\newcommand{\senh}{\operatorname{senh}}
\newcommand{\csch}{\operatorname{csch}}
\newcommand{\sech}{\operatorname{sech}}
\newcommand{\arcsec}{\operatorname{arcsec}}
\newcommand{\arcsen}{\operatorname{arcsen}}
\newcommand{\id}{\operatorname{id}}
\begin{document}




\section*{T7}

<ol>
<div class="ejercicio-box"> <h2 class="number-title"> Ejercicio</h2> Considera la función $f(x,y)=xy$. Usando solamente la definición, prueba
  que para todo punto  $(x_0,y_0)$, $f$ es diferenciable en $(x_0,y_0)$



  Hint: para la parte de la definición que involucra  límite,
  primero prueba
        $$
        f(x,y)-f(x_0,y_0)- (\partial_xf(x_0,y_0))(x-x_0)-(\partial_yf(x_0,y_0))
        (y-y_0)= (x-x_0)(y-y_0)
        $$



      <div class="ejercicio-box"> <h2 class="number-title"> Ejercicio</h2> Para cada una de las siguientes funciones calcula el gradiente.
        <ol>
        <div class="ejercicio-box"> <h2 class="number-title"> Ejercicio</h2> $f(x,y)=\frac{xy}{(x^2+y^2)^{1/2}}$
        <div class="ejercicio-box"> <h2 class="number-title"> Ejercicio</h2> $f(x,y)=\log(x^2+y^2)$
          <div class="ejercicio-box"> <h2 class="number-title"> Ejercicio</h2> $f(x,y)=\frac{x}{y}+\frac{y}{x}$
        </ol>


        
      <div class="ejercicio-box"> <h2 class="number-title"> Ejercicio</h2> Para cada una de las siguientes superficies encuentra la
        ecuación del plano tangente en el punto indicado.
        <ol>
        <div class="ejercicio-box"> <h2 class="number-title"> Ejercicio</h2> $z=x^2+y^3$ en $(1,2,9)$.
        <div class="ejercicio-box"> <h2 class="number-title"> Ejercicio</h2> $z=e^{x^2+xy}$ en $(0,1,1)$.
        <div class="ejercicio-box"> <h2 class="number-title"> Ejercicio</h2> $x^2+y^2+z^2=1$ en $(1/\sqrt{3}, 1/\sqrt{3}), 1/\sqrt{3})$.
          <div class="ejercicio-box"> <h2 class="number-title"> Ejercicio</h2> $x^2+y^2-z^2=1$ en $(1,2,-2)$.
        </ol>


                
      <div class="ejercicio-box"> <h2 class="number-title"> Ejercicio</h2> Usando la notación  $p=(x,y)$ y  $p_0=(x_0,y_0)$, prueba que el
        límite de la definición de derivada  es equivalente a
        $$
        \lim_{p\to p_0}\frac{f(p)-f(p_0)-T(p-p_0) }{\|p-p_0\|}=0.
        $$
        donde $T$ es la función lineal asociada al vector
        $(\partial_xf(x_0,y_0), \partial_yf(x_0,y_0))$.


        

                <div class="definicion-box"><h2 class="number-title">  Definición</h2>
          Sea $U$ un abierto de $\mathbb{R}^n$ y $F:U \to \mathbb{R}^m$ una función.
          Escribe las funciones coordenadas de $F$ como $F(p)=(f_1(p), \dots, f_m(p))$,
          donde cada $f_i$ es una función que toma valores en $\mathbb{R}$. Supon que
          todas las derivadas parciales de todas la $f_i$ existen. La matriz de derivadas
          parciales es la matriz de $m\times n$ cuya entrada $(i,j)$ es $\partial_{p_j}f_i$.
        </div>

      <div class="ejercicio-box"> <h2 class="number-title"> Ejercicio</h2> Para cada una de las siguientes funciones encuentra la matriz de derivadas
        parciales

        <ol>
        <div class="ejercicio-box"> <h2 class="number-title"> Ejercicio</h2> $F(x,y)=(xe^y, ye^x)$
        <div class="ejercicio-box"> <h2 class="number-title"> Ejercicio</h2> $F(x,y)=(xy\cos(x), xy\sen(y))$
        <div class="ejercicio-box"> <h2 class="number-title"> Ejercicio</h2> $F(x,y)=(xy+x^2,x^2+y^2, x^3+xy+y^3)$
        <div class="ejercicio-box"> <h2 class="number-title"> Ejercicio</h2> $F(x,y,z)=(xyz, x^2y^2z^2)$
        </ol>


        
      <div class="ejercicio-box"> <h2 class="number-title"> Ejercicio</h2> Supongamos que  $F:\mathbb{R}^n \to \mathbb{R}^m$ es una función lineal.
        Usando la definición demuestra que, para todo $p_0\in \mathbb{R}^n$, $F$ es
        diferenciable en $p_0$ y $D_{p_0}F=F$.


      <div class="ejercicio-box"> <h2 class="number-title"> Ejercicio</h2> Sea $U$ un abierto de $\mathbb{R}^n$, $p_0\in U$  y $f,g:U \to \mathbb{R}$ dos funciones diferenciables en 
         $p_0$. Demuestra que
        $$
        \nabla_{p_0}(fg)=f(p_0)\nabla_{p_0}g + g(p_0)\nabla_{p_0}f  
        $$




          <div class="ejercicio-box"> <h2 class="number-title"> Ejercicio</h2> Sea $U$ un abierto de $\mathbb{R}^2$, $p_0\in U$
            y $F:U \to \mathbb{R}^2$ una función. Escribamos
            las funciones coordenadas
            $$
            F(x,y)=(f_1(x,y), f_2(x,y)).
            $$
            Supon que las derivadas parciales de  $f_1$ y
            $f_2$ existen en $(x_0,y_0)$.
            <ol>
            <div class="ejercicio-box"> <h2 class="number-title"> Ejercicio</h2> Demuestra que, para $i=1,2$
              \begin{eqnarray*}
                & & \frac{|f_i(x,y)-f_i(x_0,y_0)-(\partial_xf_i(x_0,y_0))(x-x_0)-
                (\partial_yf_i(x_0,y_0))(y-y_0)|}{\|(x-x_0,y-y_0)\|} \\
              &\leq &
              \frac{\|F(x,y)-F(x_0,y_0)-T(x-x_0,y-y_0) \|}{\|(x-x_0,y-y_0)\|}
              \end{eqnarray*}
            donde $T$ es la matriz de darivadas parciales evaluadas
            en $(x_0,y_0)$.

            Hint: primero prueba  que, para todo vector $(a,b)$
            $$
            |a|\leq \| (a,b)\|, \quad |b|\leq \|(a,b) \|
            $$
            
          <div class="ejercicio-box"> <h2 class="number-title"> Ejercicio</h2> Usando el inciso anterior demuestra que, si suponemos
            que $F$ es diferenciable en $(x_0,y_0)$, entonces
            $f_1$ y $f_2$ son diferenciables en $(x_0,y_0)$.

          <div class="ejercicio-box"> <h2 class="number-title"> Ejercicio</h2> Demuestra que
            \begin{eqnarray*}
              & & \frac{\|F(x,y)-F(x_0,y_0)-T(x-x_0,y-y_0) \|}{\|(x-x_0,y-y_0)\|} \\
              &\leq &
                      \frac{|f_1(x,y)-f_1(x_0,y_0)-
                      (\partial_xf_1(x_0,y_0))(x-x_0)-
                      (\partial_yf_1(x_0,y_0))(y-y_0)|}
                      {\| (x-x_0,y-y_0)\|} \\
              &+&  \frac{|f_2(x,y)-f_2(x_0,y_0)-
                      (\partial_xf_2(x_0,y_0))(x-x_0)-
                      (\partial_yf_2(x_0,y_0))(y-y_0)|}
                      {\| (x-x_0,y-y_0)\|}
            \end{eqnarray*}

            Hint: primero prueba que, para todo vector $(a,b)$
            $$
            \| (a,b) \| \leq |a| + |b|.
            $$

            <div class="ejercicio-box"> <h2 class="number-title"> Ejercicio</h2> Usando el inciso anterior prueba que, si
          suponemos que $f_1$ y $f_2$ son diferenciables
          en $p_0$, entonces $F$ también es diferenciable
          en $p_0$.


          Nota: este ejercicio prueba que una función,
          con valores vectoriales, es diferenciable si y sólo si,
          todas sus funciones coordenadas son diferenciables.
         
          </ol>

        
</ol>


  
       \end{document}
